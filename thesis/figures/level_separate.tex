\begin{figure}[\placement]
\centering

%\resizebox{0.95\textwidth}{!}{
\begin{tikzpicture}

\node[circle,draw] (r) {};
\node[circle,draw] (r1) at ($ (r) + (-1.5,-1) $) {};
\node[circle,draw,right of=r1] (r2) {};
\node[right of=r2] (r3) {\dots};
\node[circle,draw,right of=r3] (rn) {};

\node (v1) at ($ (r1) + (0, -1) $) {$V_1$};
\node (v2) at ($ (r2) + (0, -1) $) {$V_2$};
\node (vn) at ($ (rn) + (0, -1) $) {$V_n$};

\drawedges{r/r1,r/r2,r/rn,r1/v1}

\end{tikzpicture}
%}

\caption[Modelling the separation by a \PT-tree]{That the sets~$V_i$ for~$i\in\range{n}$ are separated can be modelled
by a \PT-tree.}
\label{figure:level_separate}
\end{figure}