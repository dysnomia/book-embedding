%% conclusion.tex
%%

%% ==================
\chapter{Conclusion}
\label{ch:conclusion}
%% ==================

In this thesis we considered the book
embedding problem where the assignment of edges to pages has already been fixed.

We proved that \probBook is \NP-complete
for a linear number of pages in \myref{chapter:complexity}, even if the pages are matchings. 
In the same chapter we 
showed how \probBook can still be solved in super-polynomial
time by expressing it with 3-\CNF-formulae.
Though matchings are a nicely restricted case that is already \NP-complete, 
it is dissatisfying that we need an unbounded number of pages for our \NP-hardness proof.
We would like to show \NP-completeness for a constant number of pages similar to the general book embedding problem,
which is \NP-complete for two pages~\cite{Bernhart79}. The problem \probBook may be \NP-complete for the next smaller case
of three pages, but proving or disproving that seems to be quite difficult.

The remainder of the work was concerned with a variety of special cases
and restrictions of \probBook in \myref{chapter:special}. We first considered pages 
containing connected graphs and
showed that embeddability can be decided in linear time in this case by
representing all possible book embeddings using a \PQ-tree. 
%With the same approach we can reduce a general book embedding instance to a \probPQ instance.

Next, we dealt with the very opposite with regards to connectivity:
the pages are disjoint perfect matchings. We showed that bipartiteness is
necessary for embeddability in this case and provided bipartite examples and counterexamples for
all numbers of pages except for three pages. We computed that the smallest counterexample for three pages has at least~20 vertices
and at most~28. When two matchings form a cycle, we found a smallest
counterexample of order~28. This is too large
for us to be able to infer anything useful from it. One obvious extension of this case is to find some structure
in the counterexamples even though they are large and, maybe, get a better necessary
condition or a good sufficient condition. The counterexample for three pages may also yield a clue on
whether \probBook is \NP-complete for three pages.

The problem that we considered after that was to restrict the order of the vertices on the spine
by a \Q-tree. We showed that the book constraints turn into simple constraints
on the \Q-tree in this case. This allowed us to solve the problem in quadratic time. The most interesting
continuation of this line of thought is to make the restriction more in accordance with its motivation.
That is, to use \PT-trees as in the reduction of \SEFECON to a 2-page \probPTree instance by Angelini et.\,al.~\cite{angelini11}. We already argued that
this does not make the problem easier than \probBook since a \PT-tree can represent
all permutations on its leaves. Still, maybe we can get a solution for just two pages
which is all that is needed for solving \SEFECON.

Finally, we varied the book embedding problem by allowing multiple spines. We showed
that this case is equivalent to a restricted 2-page \probPTree instance. Although this did not efficiently solve the problem, it provided
us with several future extensions: 
%Provide a polynomial time algorithm for  \probPTree and get a solution of \probMul, show the \NP-com\-plete\-ness of \probMul and get the \NP-completeness of \probPTree or provide a polynomial time algorithm for \probMul and get an efficient algorithm for a special case of \probPTree. 
\begin{itemize}
\item Provide a polynomial time algorithm for  \probPTree and get an efficient solution of \probMul.
\item Prove the \NP-completeness of \probMul and get the \NP-completeness of \probPTree.
\item Provide a polynomial time algorithm for \probMul and get an efficient algorithm for a special case of \probPTree.
\end{itemize}
The last
extension may also give some
helpful pointers on how to approach the general \probPTree problem.

All in all, the most important continuation of this work is to find the computational
complexity of two problems: \probBook for
a constant number of pages and \probPTree. In the following we
list possible approaches and sub-problems that could possibly be of use, ordered decreasingly by how likely
we believe the approach to succeed or how useful the sub-problem is:

\begin{enumerate}
\item Prove the \NP-completeness of \probBook for a constant number of pages:
\begin{enumerate}
\item Compute a smallest bipartite counterexample of \probThreeMatching.
\item Show that \probThreeMatching is \NP-complete by looking at the structure of \myref{figure:two_cycles} or derive a necessary and sufficient condition from it
that is efficiently checkable.
%\item Show that \probPQ is \NP-complete for a constant number of \PQ-trees.
\end{enumerate}

\item Find the computational complexity of \probPTree:
\begin{enumerate}
\item Give an efficient algorithm for \probMul or show that it is \NP-complete.
%\item Show that \probMul is \NP-complete.
\item Solve \SEFECON.
\item Generalise the approach of \myref{section:trees} to \PT-trees.
\end{enumerate}
\end{enumerate}